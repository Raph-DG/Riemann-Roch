% In this file you should put the actual content of the blueprint.
% It will be used both by the web and the print version.
% It should *not* include the \begin{document}
%
% If you want to split the blueprint content into several files then
% the current file can be a simple sequence of \input. Otherwise It
% can start with a \section or \chapter for instance.

\begin{definition}[Dimension Function]
  \label{def:dimfun}
  A dimension function on a scheme $X$ is defined to be function $\delta : X \rightarrow \mathbb{Z}$
  satisfying such that if $y : X$ is an immediate specialisation of $x : X$, then
  $\delta(x) = \delta(y) + 1$, and $\delta$ is monotone with respect to the
  specialisation order.

  HANDLED BY FORMAL METHODS STUDENTS
\end{definition}

\begin{definition}[Catenary Space]
  We say a space is catenary if between every pair of irreducible closed subsets there
  exists a maximal chain of irreducible closed subsets, and all such chains have the
  same length.

  HANDLED BY FORMAL MATHODS CLASS
\end{definition}

\begin{definition}[Universally Catenary Scheme]
  \label{02J7}
  We say a scheme $S$ is universally catenary if, every scheme $X$ with a morhphism
  $X \rightarrow S$ locally of finite type is catenary.
\end{definition}

From now, let $(S, \delta)$ denote a locally Noetherian universally catenary scheme
with dimension funtion $\delta$.



\begin{theorem}[Lifting of dimension functions]
  \label{02JW}
  Let $f:X \rightarrow S$ be a morphism of schemes.
  Assume $f$ locally of finite type. Then the map
  \[
  \begin{split}
  \delta_{X / S} &X \rightarrow \mathbb{Z}\\
  &x \mapsto \delta (f (x)) + \text{trdeg}_{\kappa(f(x))} \kappa (x)
  \end{split}
  \]
  where $\kappa(x)$ denotes the residue field at $x$.

\end{theorem}


\begin{definition}[Algebraic Cycle on a sheme]
  \label{def:algcycle}
  \leanok
  \lean{AlgebraicCycle}
  An algebraic cycle on a scheme $X$ is defined to be a function from
  $X \rightarrow \mathbb{Z}$ with locally finite support.
\end{definition}

\begin{definition}[Pushforward of cycles]
  \label{def:pushcycle}
  \leanok
  \lean{AlgebraicCycle.map}



\end{definition}


For the definitions about divisors, let $X$ be a noetherian integral scheme
which is regular in codimension 1 meaning all local rings of codimension 1 are
regular (where regular means minimal number of generators for maximal ideal is
equal to Krull dimension).

\begin{definition}\label{def:primediv}
  A prime divisor on $X$ is an irreducible subvariety of $X$ with codimension $1$.
  \end{definition}

\begin{definition}\label{def:weil}
  A Weil Divisor on $X$ is a formal $\mathbb{Z}$-linear combination of prime divisors of $X$. We denote the free abelian group consisting of all the Weil divisors by $\Div(X)$.
\end{definition}

\begin{definition}[Hartshorne p.140]\label{def:constratsheaf}
  Let $X$ be a scheme. For each open affine subset $U = Spec A$,
  let $S$ be the set of elements of $A$ which are not zero divisors,
  and let $K(U)$ be the localization of $A$ by the multiplicative system $S$.

  We call $K(U)$ the total quotient ring of $A$.
  For each open set $U$, let $S(U)$ denote the set of elements of
  $\Gamma(U, \reg_X)$ which are not zero divisors in each local ring $\reg_x$ for $x \in U$.
  Then the rings $S(U)^{-1} \Gamma(U,\reg_X)$ form a presheaf,
  whose associated sheaf of rings $X$ we call the sheaf of total quotient rings
  of $\reg$. On an arbitrary scheme, the sheaf $\mathscr{K}$ replaces the concept
  of function field of an integral scheme.

  We denote by $\mathscr{K}^*$ the sheaf (of multiplicative groups)
  of invertible elements in the sheaf of rings $\mathscr{K}$.

  Similarly $\reg^*$ is the sheaf of invertible elements in $\reg$.
\end{definition}

\begin{definition}[Hartshorne p.141]\label{def:cartierdivisor}
A Cartier divisor is a global section of the sheaf $\mathscr{K}^* / \reg^*$.
A Cartier divisor is principal if it is in the image of the natural map
$\Gamma(X,\mathscr{K}) \rightarrow T(X, \mathscr{K}^* / \reg^*)$.

Two Cartier divisors are linearly equivalent if their difference is principal.
(Although the group operation on $\mathscr{K}^* / \reg^*$ is multiplication,
we will use the language of additive groups when speaking of Cartier divisors,
so as to preserve the analogy with Weil divisors.)
\end{definition}

NEED LINE BUNDLE DIVISOR EQUIVALENCE - I.E. LINE BUNDLE
ASSOCIATED WITH A DIVISOR SO WE CAN DO

NEED AN EQUIVALENCE OF DIVISOR DEFINITIONS BECAUSE LINE BUNDLE
DIVISOR EQUIVALENCE REALLY NEEDS CARTIER BUT DOING INDUCTION BY
ADDING POINTS TO THE DIVISOR IS REALLY USING A WEIL DIVISOR


NEED A COHOMOLOGY DEFINITION

NEED:-
- COHOMOLOGY LONG EXACT SEQUENCE FOR EULER CHAR

- NEED TENSOR PRODUCT OF SHEAVES AND
THAT TENSORING WITH LINE BUNDLES IS EXACT

-



\begin{lemma}[Skyscraper Sheaf has Vanishing Higher Cohomology]\label{lem:skyscrapervanishing}
Given a skyscraper sheaf $\mathscr{F}$ on a topological space $Y$,
for all $i \in \mathbb{N} \setminus \{0\}$, $H^i(\mathscr{F}, Y) = 0$.
\end{lemma}
\begin{proof}
Hartshorne Chap 3 Prop 2.5 p.208 shows this by showing skscraper sheaves are flasque.
I think I prefer showing that given any cover, can always refine it so only one set
in the cover contains the point where the sheaf is supported. Hence the sheaf
has no sections on intersections of sheaves of the cover and so neessarily has
vanishing higher cohomology.
\end{proof}

\begin{theorem}[Euler Characteristic Addititive] \label{thm:eulercharadditive}
  Given a short exact sequence of sheaves:
  \[
  0 \rightarrow \mathscr{F}' \rightarrow \mathscr{F} \rightarrow \mathscr{F}'' \rightarrow 0,
  \]
  the Euler characteristic is additive, that is:
  \[
  \chi(\mathscr{F}) = \chi(\mathscr{F}') + \chi(\mathscr{F}'').
  \]
\end{theorem}
  \begin{proof}[Proof sketch]
    This follows by first taking the corresponding long exact sequence of cohomology groups
    \[
    \begin{split}
    0 &\xrightarrow{\varphi_0'} H^0(X, \mathscr{F'}) \xrightarrow{\varphi_0}  H^0(X, \mathscr{F}) \xrightarrow{\varphi_0''}  H^0(X, \mathscr{F''})\\
    &\xrightarrow{\varphi_1'} H^1(X, \mathscr{F'}) \xrightarrow{\varphi_1} H^1(X, \mathscr{F}) \xrightarrow{\varphi_1''} H^1(X, \mathscr{F''}) \xrightarrow{\varphi_2'} ....\\
    \end{split}
    \]
    We then see that this splits as the following set of short exact sequences:
    \[
    0 \rightarrow \coker(\varphi_p') \xrightarrow{\varphi_p} H^p(X, \mathscr{F}) \xrightarrow{\varphi_p''} \im(\varphi_p'') \rightarrow 0.
    \]
    But since these are vector spaces, the dimension on these short exact sequences is additive, and so the alternating sum of the dimensions of our cohomology groups is $0$. Rearranging this identity then gives the desired result.
\end{proof}

\begin{theorem}[Riemann-Roch for curves]\label{thm:riemannroch}
  Given a smooth projective curve $X$ over an algebraically closed field $k$ and a divisor $D$ on $X$, the following identity holds:
  \[
  \chi(\reg_X(D)) = \chi(\reg_X) + \deg D
  \]

  In this context, the word divisor here is implying that what definition we use
  doesn't make a difference (which is mainly where the smooth and projective
  assumptions are actually used).

  For this, we probably want to cover some theory about divisors and line bundles
  in here.
  \end{theorem}
  \begin{proof}
  We will prove this by an induction argument on $D$ noting that any divisor $D$ can be built up by adding or subtracting points $P$ starting from the divisor $0$. To that end, we will first note that $\chi(\reg_X(0)) = \chi(\reg_X) + 0$, hence our formula holds for the $0$ divisor. Now, for our inductive case, suppose our formula holds for a divisor $D$, i.e. that $\chi(\reg_X(D)) = \chi(\reg_X) + \deg D$. We'll try to show it must then hold for $D + P$, where $P$ is the divisor of a point. Then, we recall that we have the following exact sequence:

  \[
  0 \rightarrow \reg_X(-P) \rightarrow \reg_X \rightarrow \reg_P \rightarrow 0.
  \]
  Now, we can twist this exact sequence by $\reg_X(D+P)$, noting that since $\reg_X(D+P)$ is a line bundle, $\reg_X(D+P) \otimes_{\reg_X} \reg_P \simeq \reg_X \otimes_{\reg_X} \reg_P \simeq \reg_P$. So, we end up with the following exact sequence:

  \[
  0 \rightarrow \reg_X(D) \rightarrow \reg_X(D + P) \rightarrow \reg_P \rightarrow 0
  \]

  Hence, we have by \autoref{thm:eulercharadditive} that $\chi(\reg_X(D+P)) = \chi(\reg_X(D)) + \chi(\reg_{X, p})$. Now, we know from \autoref{lem:skyscrapervanishing} that $\chi(\reg_{X, p}) = 1$, hence:
  \[
  \begin{split}
  \chi(\reg_X(D+P)) &= \chi(\reg_X(D)) + 1,\\
  &= \chi(\reg_X) + \deg(D) + 1, \text{ by inductive hypothesis,}\\
  &= \chi(\reg_X) + \deg(D + P).\\
  \end{split}
  \]
  Now, we have another inductive case where we know our property holds for some divisor $D$, and we need to show it holds for $D - P$. The argument for this case is very similar, instead twisting our initial exact sequence by $\reg_X(D)$ to get:
  \[
  0 \rightarrow \reg_X(D-P) \rightarrow \reg_X(D) \rightarrow \reg_P \rightarrow 0.
  \]
  Then, again using the additivity of the Euler characteristic, we get:
  \[
  \begin{split}
  \chi(\reg_X(D-P)) &= \chi(\reg_X(D)) -1,\\
  &= \chi(\reg_X) + \deg(D) - 1, \text{ by inductive hypothesis,}\\
  &= \chi(\reg_X) + \deg(D - P).\\
  \end{split}
  \]
  Thus, by induction, our property is proven.
  \end{proof}
